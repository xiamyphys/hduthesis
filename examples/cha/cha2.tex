\chapter{概述}

\section{激光打靶系统概述}

激光打靶系统\cite{cn1,cn2,cn3}的工作原理是采用激光脉冲来模拟枪弹的射击,该系统一般包括激光发射部分、激光信号检测模块、打靶成绩处理和显示部分。如图 2-1 所示,当射手瞄准完毕扣动扳机时,半导体激光器会发出激光脉冲,射向目标上的光电探测器,如果击中目标,则激光脉冲被光电探测器接收并转换为电信号,经电路处理能识别射击的弹着点,信号经处理编码后传输到计算机。

\begin{figure}[htbp]
  \centering
  \begin{tikzpicture}
  [
    every node/.style = { font = \small, inner sep = 1ex }
  ]
    \node (a) [draw, rectangle] at (0,0) {光电探测器};
    \node (b) [draw, rectangle] at (3.5,0) {信号处理电路};
    \node (c) [draw, rectangle] at (7,0) {计算机处理};
    \node (d) [draw, rectangle] at (0,-1.5) {半导体激光器};
    \node (e) [draw, rectangle] at (4,-1.5) {激光枪扳机};
    \draw [->, thick] (a.east) -- (b.west);
    \draw [->, thick] (b.east) -- (c.west);
    \draw [->, thick] (d.north) -- (a.south);
    \draw [->, thick] (e.west) -- (d.east);
  \end{tikzpicture}
  \caption{激光打靶系统原理图}
\end{figure}

半导体激光器\cite{cn4,en5}一般平行地安装在武器装备的枪管、炮管或导弹发射架上,它可以发射一束与武器射击方向一致的激光脉冲。目前的激光器一般都采用半导体激光器,因为这种激光器的输出功率低,不会伤害眼睛,而且效率高、功耗小,不但可以摆脱大而重的电源设备,激光器本身也可以制作得很小、很轻。光电探测器\cite{en6}具有射击靶的形状,可以是点探测器和面探测器,通常数量较多,构成多个信号检测通路。根据光电探测器的响应位置来判断激光射击击中的靶位。

激光打靶采用以光代弹的形式进行射击训练,是激光武器模拟器中最常见的一种。最初的激光打靶系统只能进行瞄准射击训练,随着计算机和微处理器技术的发展,其用途扩大到可进行多种武器的模拟训练。随着研究和探索的深入,激光打靶系统的功能将进一步完善,能够更接近于武器装备在实际使用中的表现,增强真实感。同时,通过与电子技术相结合,进一步提高激光模拟的自动化、智能化水平。

激光武器模拟器有以下几个方面的发展趋势:

\begin{enumerate}
  \item 可以模拟的武器越来越多,激光武器模拟器正朝着系列化、组件化的方向发展,一个基本的激光射击模拟器只要稍加改动就可适用于其他武器系统。系列化、组件化的好处是便于使用、更换和维修,同时价格也便宜。
  \item 从激光射击模拟器向激光交战模拟器发展,先进的激光交战模拟器能使坦克、战斗车辆、反坦克武器等有机的结合在一起进行训练,每部兵器既是攻击者,又是被攻击者,完全模仿实战中的作战环境,不仅能提高战士使用武器的技能,还可以教会他们如何在战争中保护自己。
  \item 采用各种新技术增加模拟的逼真性,例如用计算机来记录、控制整个训练演习的进程,评定战士在演习中的表现等。
\end{enumerate}

\section{本设计方案思路}

本设计以实现信号的良好检测和数据转换、传输为主要目的;以信号检测,信号编码和数据传输为主要设计内容。

在信号检测方面设计单脉冲小信号的放大电路和信号整形电路;在信号编码方面,要解决多路信号的编码问题,还要考虑到编码的优先选择问题;在脱靶问题的处理方法上,对打靶和信号采集传送进行同步化处理(详见第二章的硬件设计部分),把脱靶的情况与中靶的情况归为一类处理;数据传输采用 UART 串口通信。

\section{研发方向和技术关键}

\begin{enumerate}
  \item 合理划分激光靶的光电探测器,提高系统的精度;
  \item 单脉冲小信号的放大和整形;
  \item 多路优先编码器的扩展;
  \item 与微机进行数据传输,方便成绩的统计、保存、显示和查询。
\end{enumerate}

\section{主要技术指标}

\begin{enumerate}
  \item \makebox[9\ccwd][l]{激光脉宽:}大于$\qty1\ms$
  \item \makebox[9\ccwd][l]{激光脉冲响应幅度:}约$\qty{10}\mV$
  \item \makebox[9\ccwd][l]{打靶距离:}$\qty{30}\m$
  \item \makebox[9\ccwd][l]{串行输出帧格式:}射击次数\ensuremath+所击中的光电探测器的编号
\end{enumerate}